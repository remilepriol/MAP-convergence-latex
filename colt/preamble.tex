% \usepackage[round]{natbib}
\renewcommand{\bibname}{References}
\renewcommand{\bibsection}{\subsubsection*{\bibname}}
\usepackage[rightcaption]{sidecap}

%\usepackage{microtype} % RLP tried it, no change.
%\usepackage[utf8]{inputenc} % allow utf-8 input
\usepackage[T1]{fontenc}    % use 8-bit T1 fonts
\usepackage{lmodern}
\usepackage{xcolor}
\usepackage{url}            % simple URL typesetting
\usepackage{booktabs}       % professional-quality tables
\usepackage{microtype}      % microtypography
\usepackage{graphicx}
\usepackage{floatrow}
%\usepackage{titlesec}

%\titleclass{\part}{top}
%\titleformat{\part}[display]
%{\normalfont\huge\bfseries}{\centering\partname\ \thepart}{20pt}{\Huge\centering}
%\titlespacing*{\part}{0pt}{50pt}{40pt}

\graphicspath{{figs/}}
%\usepackage{subfigure}
%\usepackage{subcaption}
%\usepackage{placeins}
\usepackage{hyperref}       % hyperlinks

\usepackage{array}
\usepackage{wrapfig}

\hypersetup{ % SLJ: my standard paper setup...
	pdftitle={MAP convergence},
	pdfkeywords={},
	pdfborder=0 0 0,
	pdfpagemode=UseNone,
	colorlinks=true,
	linkcolor=blue, %mydarkblue,
	citecolor=blue, %mydarkblue,
	filecolor=blue, %mydarkblue,
	urlcolor=blue, %mydarkblue,
	pdfview=FitH,
	pdfauthor={Anonymous},
	% draft,
	final,
}

% To make a table of contents only for the appendix
\usepackage[toc,page,header]{appendix}
\usepackage{minitoc}
\renewcommand \thepart{}
\renewcommand \partname{}

% To make colored boxes
\usepackage[framemethod=TikZ]{mdframed}
\newmdenv[
	roundcorner=10pt,
	backgroundcolor=Red!30,
]{important}


\usepackage[capitalise]{cleveref}
\newcommand{\rlp}[1]{\textcolor{BrickRed}{(RLP:#1)}}
\newcommand{\fdk}[1]{\textcolor{Periwinkle}{(fdk:#1)}}
\newcommand{\TODO}[1]{\textcolor{cyan}{(TODO #1)}}
\newcommand{\tocite}{\textcolor{purple}{(add citation)}}

% my packages
\usepackage{math_commands}
% some custom math commands
% \newtheorem{proposition}{Proposition}
\newtheorem{problem}{Open Problem}

\newcommand*{\expect}[2][]{\ensuremath{\mathbb{E}_{#1} \left[ #2 \right] }} % expectation operator
\newcommand*{\expecti}[2][]{\ensuremath{\mathbb{E}_{#1} [ #2 ] }} % expectation operator

\newcommand{\cond}{\,\vert\,}
\newcommand{\logpart}{A}
\newcommand{\conj}{{\logpart^*}}
\newcommand{\bregman}{\cB_\logpart}
\newcommand{\bregmanconj}{\cB_{\logpart^*}}
\newcommand{\nat}{\theta}
\newcommand{\m}{\mu}
\newcommand{\meanp}{\m}
\newcommand{\decrement}{D}
\newcommand{\linear}{\ell} % linearization of a function
\newcommand{\lr}{\gamma} % learning rate, or step-size
\newcommand{\lin}[1]{\left\langle#1\right\rangle}

\newcommand{\MAPm}{\hat \m_n}
\newcommand{\MAPt}{\hat \nat_n}
\DeclareMathSymbol{\shortminus}{\mathbin}{AMSa}{"39}

\newcommand{\stgcvx}{\alpha} % strong convexity
\newcommand{\smooth}{\beta} % smoothness

\newcommand*{\greencmark}{\textcolor{green}{\cmark}}
\newcommand*{\redxmark}{\textcolor{red}{\xmark}}

\setcounter{tocdepth}{2}